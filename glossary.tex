
% From https://www.overleaf.com/learn/latex/Glossaries

\makeglossaries % Prepare for adding glossary entries


%\newglossaryentry{ps}
%{
%    name=Pipeline Shell,
%    description=Shell responsible for the cycle-accurate timing of the values generated by the \acrshort{iss}
%}
%
%\newglossaryentry{stagebased}
%{
%    name=stage-based pipeline simulation,
%    description=
%}
%
%\newglossaryentry{timewheel}
%{
%    name=Cycle-based time wheel simulation,
%    description=
%}
%
%\newglossaryentry{cv32s}
%{
%    name=CV32E40S,
%    description=An open source RISC-V processor from the OpenHW Group.
%}
%
%\newglossaryentry{cv32x}
%{
%    name=CV32E40X,
%    description=An open source RISC-V processor from the OpenHW Group.
%}
%
%\newglossaryentry{core-v-verif}
%{
%    name=core-v-verif,
%    description=An open source RISC-V processor from the OpenHW Group.
%}
%
%
%\newglossaryentry{flush}
%{
%    name=Flush,
%    description=To clear uncommited instructions from the pipeline. 
%}
%
%\newglossaryentry{commited}
%{
%    name=Commited,
%    description=When the results of an instruction is written back to the register file in the Writeback stage of the pipeline.
%}
%
%\newglossaryentry{retired}
%{
%    name=Retired,
%    description=When an instruction is \gls{commited} after being in the WB stage and is no longer in the pipeline.
%}
%
%\newglossaryentry{step}
%{
%    name=Step,
%    description=When the ISS executes one instruction\, or the instructions move to the next pipeline stage.
%}
%
%\newglossaryentry{checkpoint}{
%    name=checkpoint,
%    description=Points along the execution the processor can revert to\, where the state of the processor has been stored.
%}

% --------------------
% ----- Acronyms -----
% --------------------

\newacronym{riscv}{RISC-V}{Reduced Instruction Set Computer (RISC) Five}
\newacronym{iss}{ISS}{Instruction Set Simulator}
\newacronym{uvm}{UVM}{Universal Verification Methodology}
\newacronym{dut}{DUT}{Device Under Test}
\newacronym{fev}{FEV}{Formal Equivalence Verification}
\newacronym{rtl}{RTL}{Register Transfer Level}
\newacronym{isa}{ISA}{Instruction Set Architecture}
\newacronym{rm}{RM}{Reference Model}
\newacronym{sva}{SVA}{SystemVerilog Assertions}
\newacronym{fv}{FV}{Formal Verification}
\newacronym{rvfi}{RVFI}{RISC-V Formal Interface}
\newacronym{rvvi}{RVVI}{RISC-V Verification Interface}
\newacronym{bsp}{BSP}{Board Support Package}
\newacronym{gpr}{GPR}{General purpose registers}
\newacronym{pc}{PC}{Program Counter}
\newacronym{if}{IF}{Instruction Fetch}
\newacronym{id}{ID}{Instruction Decode}
\newacronym{ex}{EX}{Execute}
\newacronym{wb}{WB}{Write-back}
\newacronym{lsu}{LSU}{Load Store Unit}
\newacronym{csr}{CSR}{Control and Status Register}
\newacronym{vip}{VIP}{Verification IP}
\newacronym{nmi}{NMI}{Non Maskable Interrupts}
\newacronym{dsa}{DSA}{domain-specific architectures}
\newacronym{dm}{DM}{Debug Module}
\newacronym{clic}{CLIC}{Core-Local Interrupt Controller}
\newacronym{clint}{CLINT}{Core Local Interrupt Controller}
\newacronym{dpi}{DPI}{Direct Programming Interface}
\newacronym{abv}{ABV}{Assertion-Based Verification}
\newacronym{obi}{OBI}{Open Bus Interface}



