\chapter{Processor::step() from Proc.cc}
\label{app:spike_step}


\begin{clisting}
st_rvfi Processor::step(size_t n, st_rvfi reference) {
  st_rvfi rvfi;
  memset(&rvfi, 0, sizeof(st_rvfi));
  commit_log_reg_t prev_commit_log_reg = this->get_state()->log_reg_write;


  this->taken_trap = false;
  this->which_trap = 0;

  // Store the state before stepping
  state_t prev_state = *this->get_state();

  // Disable WFI to handle the timing outside of spike.
  in_wfi = false;
  processor_t::step(n);

  rvfi.pc_rdata = this->last_pc;

  // Add overwritten values from memory writes during the step
  prev_changes_t prev_changes(prev_state, this->get_state()->log_mem_pre_write);
  if(max_previous_states > 0) {
    previous_states.push_front(prev_changes);
  }
  if(previous_states.size() > max_previous_states) {
    previous_states.pop_back();
  }

  rvfi.pc_wdata = this->get_state()->pc; // Next predicted PC

  rvfi.mode = this->get_state()->last_inst_priv;
  rvfi.insn =
      (uint32_t)(this->get_state()->last_inst_fetched.bits() & 0xffffffffULL);

  // TODO FIXME Handle multiple/zero writes in a single insn.
  auto &reg_commits = this->get_state()->log_reg_write;
  auto &mem_write_commits = this->get_state()->log_mem_write;
  auto &mem_read_commits = this->get_state()->log_mem_read;
  int xlen = this->get_state()->last_inst_xlen;
  int flen = this->get_state()->last_inst_flen;

  rvfi.rs1_addr = this->get_state()->last_inst_fetched.rs1();
  // TODO add rs1_value
  rvfi.rs2_addr = this->get_state()->last_inst_fetched.rs2();
  // TODO add rs2_value


  if(this->next_rvfi_intr){
    rvfi.intr = next_rvfi_intr;
    this->next_rvfi_intr = 0;

    //Add csr changes that happened during first interrupt step
    reg_commits.insert(prev_commit_log_reg.begin(), prev_commit_log_reg.end());
  }
  
  // Output dbg caused from EBREAK the previous instruction
  if(this->next_debug) {
    rvfi.dbg = this->next_debug;
    this->next_debug = 0;
  }

  if(this->state.debug_mode  && (prev_state.debug_mode == 0)){
    // New external debug request
    if((this->halt_request != HR_NONE)){ 
      rvfi.dbg = this->get_state()->dcsr->cause;
      rvfi.dbg_mode = 1;

    // EBREAK
    } else if(this->get_state()->dcsr->cause == DCSR_CAUSE_SWBP) {
      // EBREAK will set debug_mode to 1, but we should report this at the next instruction
      rvfi.trap |= 1 << 0; //trap [0]
      rvfi.trap |= 1 << 2; //debug [2]
      rvfi.trap |= 0xE00 & ((DCSR_CAUSE_SWBP) << 9); //debug cause [11:9]
      
      this->next_debug = DCSR_CAUSE_SWBP;
    }
  }

  // Set dbg_mode to 1 the first instruction in debug mode, but delay turning 
  // off dbg_mode to the next instruction after turning off to keep dbg_mode on for dret
  if( (this->halt_request != HR_NONE)  && (prev_state.debug_mode == 0)) {
    rvfi.dbg_mode = 1;
  } else {
    rvfi.dbg_mode = prev_state.debug_mode;
  }


  if(this->taken_trap) {
    //interrrupts are marked with the msb high in which_trap
    if(this->which_trap & ((reg_t)1 << (isa->get_max_xlen() - 1))) { 
      //Since spike steps two times to take an interrupt, we store the intr value to the next step to return with rvfi
      this->next_rvfi_intr |= 1 << 0; //intr [0]
      this->next_rvfi_intr |= 1 << 2; //interrupt [2]
      this->next_rvfi_intr |= 0x3FF8 & ((this->which_trap & 0xFF) << 3); //cause[13:3]
    } else{
      rvfi.trap |= 1 << 0; //trap [0]
      rvfi.trap |= 1 << 1; //exception [1]
      rvfi.trap |= 0x1F8 & ((this->which_trap) << 3); //exception_cause [8:3]
      //TODO:
      //debug_cause     [11:9] debug cause
      //cause_type      [13:12]
      //clicptr         [14]  CLIC interrupt pending
      this->next_rvfi_intr = rvfi.trap; //store value to return with rvfi.intr on the next step
    }
  }

  uint64_t last_rd_addr = 0;
  uint64_t last_rd_wdata = 0;
  bool got_commit = false;
  for (auto &reg : reg_commits) {
    if((reg.first & 0xf) == 0x4) { //If CSR
      for (size_t i = 0; i < CSR_SIZE; i++) {
          if (!rvfi.csr_valid[i]) {
              rvfi.csr_valid[i] = 1;
              rvfi.csr_addr[i] = reg.first >> 4;
              rvfi.csr_wdata[i] = reg.second.v[0];
              rvfi.csr_wmask[i] = -1;
              break;
          }
      }
    }
    else {
      if (got_commit) {
        last_rd_addr = reg.first >> 4;
        last_rd_wdata = reg.second.v[0];
        continue;
      }
      // TODO FIXME Take into account the XLEN/FLEN for int/FP values.
      rvfi.rd1_addr = reg.first >> 4;
      rvfi.rd1_wdata = reg.second.v[0];
      // TODO FIXME Handle multiple register commits per cycle.
      // TODO FIXME This must be handled on the RVFI side as well.
      got_commit = true; // Only latch first commit
    }
  }
  // popret(z) should return rd1_addr = 0 instead of the SP to match with the cv32e40s core
  if (((this->get_state()->last_inst_fetched.bits() & MASK_CM_POPRET) == MATCH_CM_POPRET) ||
      ((this->get_state()->last_inst_fetched.bits() & MASK_CM_POPRETZ) == MATCH_CM_POPRETZ)) {
    rvfi.rd1_addr = 0;
    rvfi.rd1_wdata = 0;
  }

  bool mem_access = false; // TODO: support multiple memory writes/reads
  int read_len;
  for (auto &mem : mem_read_commits) {
    //mem format: (addr, 0, size) (value is not stored for reads, but should be the same as rd)
    if(!mem_access) {
      rvfi.mem_addr = std::get<0>(mem);
      if ((this->get_state()->last_inst_fetched.bits() & MASK_CM_POP) == MATCH_CM_POP         ||
          (this->get_state()->last_inst_fetched.bits() & MASK_CM_POPRET) == MATCH_CM_POPRET   ||
          (this->get_state()->last_inst_fetched.bits() & MASK_CM_POPRETZ) == MATCH_CM_POPRETZ ){    
        rvfi.mem_rdata = last_rd_wdata; // During pop, rd1 returns sp, so instead return value read from memory 
      } else {
        rvfi.mem_rdata = rvfi.rd1_wdata; 
      }
      //mem_rmask should hold a bitmask of which bytes in mem_rdata contain valid data
      read_len = std::get<2>(mem);
      rvfi.mem_rmask = (1 << read_len) - 1;
      mem_access = true;
    }
  }

  int write_len;
  for (auto &mem : mem_write_commits) {
    //mem format: (addr, value, size)
    if(!mem_access) {
      rvfi.mem_addr = std::get<0>(mem);
      rvfi.mem_wdata = std::get<1>(mem); // value
      //mem_wmask should hold a bitmask of which bytes in mem_wdata contain valid data
      write_len = std::get<2>(mem);
      rvfi.mem_wmask = (1 << write_len) - 1;
      mem_access = true;
    }

  }
  
  if (csr_counters_injection) {
    // Inject values comming from the reference
    if ((rvfi.insn & MASK_CSRRS) == MATCH_CSRRS) {
      if (rvfi.rs1_addr == 0) {
        reg_t read_csr = this->get_state()->last_inst_fetched.csr();
        switch (read_csr) {
        case 0xC00: // cycle
        case 0xC80: // cycleh
        case 0xB00: // mcycle
        case 0xB80: // mcycleh
          this->set_XPR(reference.rd1_addr, reference.rd1_wdata);
          rvfi.rd1_wdata = reference.rd1_wdata;
          break;
        default:
          break;
        }
      }
    }
  }

  // Remove sign extension applied by Spike in 32b mode.
  if (this->get_xlen() == 32) {
    rvfi.pc_rdata &= 0xffffffffULL;
    rvfi.rd1_wdata &= 0xffffffffULL;
  }
  return rvfi;
}

\end{clisting}
