\chapter*{Abstract}

Processor verification is a crucial aspect of processor development, consuming a significant portion of the overall development time.
Asynchronous events like interrupts and debug requests pose a significant challenge in processor verification, often leading to elusive bugs. 
\textit{Instruction Set Simulators (ISS)} are often used as a Reference Model to verify the correctness of a processor's execution and are often sufficient in normal execution. 
However, due to their instruction-level abstraction level, ISSs can not accurately simulate the timing of asynchronous events and side effects compared to the processor core. Therefore, a more advanced Reference Model that accurately simulates the timing and correctness of asynchronous events is needed.


This thesis explores and compares different approaches to implementing a reference model for open-source RISC-V processor cores that can accurately simulate asynchronous events at the cycle level. We propose a reference model architecture combining a "Pipeline Shell" and an existing ISS. The pipeline shell is responsible for modeling the timing of the pipeline and the behavior of asynchronous events, while the ISS is responsible for the functional execution of the instructions. The reference model is implemented in SystemVerilog and integrated into the OpenHW Group's verification environment for the CV32E40S core by Silicon Labs. We also focus on making the reference model compatible with formal verification.

The results show that the reference model can correctly simulate asynchronous events in many different scenarios. It has a smaller verification gap compared to using a traditional ISS as a reference model and is likely to find bugs that other reference model solutions can not. The ISS used is not compatible with formal verification. Still, the rest of the design is believed to be compatible with formal verification if the ISS is replaced with a synthesizable ISS. However, the current implementation has some limitations. The reference model is currently dependent on the DUT core for some pipeline timing details, and the complexity of the pipeline shell has led to multiple unresolved bugs.







