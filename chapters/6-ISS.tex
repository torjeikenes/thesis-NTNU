\chapter{Instruction Set Simulator (ISS)}
\label{ch:ISS}

%\tmp{Burde jeg delt opp kapittelet tydeligere i: 1. Adding CV32E40S support to spike, og 2. Modifications to support pipeline shell??}

The \acrfull{iss} is a critical component in our reference model, responsible for the functional execution of each instruction While the pipeline shell handles the timing and core-specific pipeline behaviour, the ISS ensures the correct interpretation and execution of the instructions according to the RISC-V ISA specification \cite{watermanRISCVInstructionSet2019, watermanRISCVInstructionSet2021}.

This chapter will first discuss the requirements of the ISS, and then discuss the selection and modification of the ISS to meet the requirements. 
We will cover the following aspects:

\begin{itemize}
    \item Choosing the ISS.
    \item Adapting the ISS to the CV32E40S processor core.
    \item Modifying the ISS to properly interact with the pipeline shell.
\end{itemize}

\section{ISS Requirements}

To achieve these goals, we define the following requirements for the ISS. These requirements are collected from the previous chapters and referred to as \textbf{ISS Requirement X}, and are required for the ISS to be used as a part of the reference model. This chapter will use these requirements to discuss the choice of ISS and its necessary modifications. 

\begin{enumerate}

%\item \textbf{Provide visibility into multiple instructions} \label{issReq:visibility}
%
%\par To simulate the pipeline, we need to track multiple instructions simultaneously, corresponding to the instructions in the different pipeline stages. This requires that we can "see" the $n$ coming instructions according to the pipeline stages, or allow separation of fetching, decoding, execution, etc into separate cycles. 

\item \textbf{The ISS should be easily replaceable in the future.} \label[issReq]{issReq:replacable}

\par The ISS should be mature and thoroughly tested. It should be actively maintained and likely to be updated in the future.

%\item \textbf{The ISS should be core-independent} \label[issReq]{issReq:coreIndependent}
%
%\item[] The ISS should contain as little core-dependent functionality as possible to make the reference model easily modifiable to different cores.

\item \textbf{Modify ISS to match the core.} \label[issReq]{issReq:csr}

\item[]  The ISS should be able to be modified to match the core. Specifically regarding memory mapping and CSRs.

\item \textbf{Run \acrshort{riscv} binaries generated from core-v-verif.} \label[issReq]{issReq:binaries}

\item[]  The ISS must be able to run the same binaries as the core.

\item \textbf{Support stepping through the execution at the Instruction Level} \label[issReq]{issReq:step}

\item[]  Since the reference model runs in lock-step with the DUT, we need to be able to step through the instructions at the instruction level, and be able to only run one instruction at a time, before waiting to step into the next instruction.

\item \textbf{Provide processor state after each instruction} \label[issReq]{issReq:state}

\par The ISS needs to output the state changes executed after each instruction so that the pipeline shell can apply these changes in a cycle-accurate manner. 
%To be able to use the ISS in verification, we need to access the processor state changes after each instruction to compare with the DUT. This should preferably be in the RVFI format or enough information should be available to convert into an RVFI item.


%\item \textbf{Support modification of instructions} \label[issReq]{issReq:modification}
%
%\par To simulate interrupts in the pipeline, it might be necessary to simulate pipeline flushing and halting. To do this we should be able to modify the instructions in the pipeline e.g. to replace them with empty NOP operations to flush the pipeline or halt execution of an instruction. 

\item \textbf{Enable external injection of interrupts} \label[issReq]{issReq:interrupt} 

\par External interrupts and debug requests are inserted into the reference model independently of the ISS. To simulate the interrupts, we must be able to tell the ISS to take an interrupt by jumping to the trap handler's PC and correctly executing the side effects at the correct cycle. 

\item \textbf{The ISS should decide which interrupt to take, when multiple interrupts are enabled} \label[issReq]{issReq:interruptOrdering} 

\par As discussed in \Cref{sec:ps_interrupt}, which interrupt to take should be decided in the ISS instead of the pipeline shell. This means that the ISS must also know the interrupt priorities of the custom interrupts in bit 16 and above, in addition to the priorities of bit 15:0, specified in the RISC-V specification.

\item \textbf{Enable and disable interrupt independently of the \rv{mip} bits} \label[issReq]{issReq:interruptEnabled} 

\par The \rv{mip} bits should be able to be set without necessarily triggering an interrupt.

%\item \textbf{Configurable with different extextions} \label[issReq]{issReq:custom}
%
%\par A large driving force for using RISC-V is the ability to add custom extensions. Therefore, custom instructions and extensions must be easily created and added to the ISS without modifying large parts of the model.

%\item \textbf{Ensure maintainability while meeting requirements} \label[issReq]{issReq:update}
%
%\par The ISS should be easily maintained and updated while still meeting the requirements. Modifications to the ISS should be kept to a minimum and be done in a way that allows the ISS to be easily updated.

%\item \textbf{Actively maintained and thoroughly tested} \label[issReq]{issReq:maintained}

%\par The ISS should be mature and thoroughly tested. It should be actively maintained and likely to be updated in the future.


\item \textbf{Set interrupt dependent CSRs from outside of the ISS.} \label[issReq]{issReq:set_CSR}

\par We should be able to apply the effects of the \rv{mie} and \rv{mstatus} CSRs from outside of the ISS, to ensure interrupts are timed correctly.


\item \textbf{Output CSRs relevant for interrupt timing.} \label[issReq]{issReq:CSR_out}

\par The ISS should output the CSRs that have an influence on interrupt timing, so that these can be applied at the correct time. This includes the \rv{mstatus.mie} and \rv{mie} CSRs.

\item \textbf{Revert the state of the ISS a variable number of instructions} \label[issReq]{issReq:revertState}

\par To discard instructions flushed in the pipeline, we must be able to revert the ISS to a previous state, as discussed in \Cref{sec:ps_revertion}. We must be able to revert the \acrshort{pc}, \acrshort{gpr}, \acrshort{csr}, and memory operations a set number of steps.


\item \textbf{[optional] The ISS should compatible with formal verification} \label[issReq]{issReq:formal}

\par The ISS should be compatible with formal verification by being able to compile into synthesizable RTL code.
%\item \textbf{Load the same instructions as the DUT} \label[issReq]{issReq:DUTins}

%\par The ISS should be able to load the same instructions as the DUT.

%\par As discussed in \cref{sec:shell_iss_interaction}, to be able to model the reference model around an ISS, it is helpful to be able to separate the register file from the rest of the ISS, or be able to have a copy of the register file, and update the register file inside the ISS when necessary.

\end{enumerate}

\section{Choosing ISS}

The specialization project discussed multiple potential \acrshort{iss}s (\Cref{sec:pw_iss}) and found that Spike \cite{SpikeRISCVISA2023} and sail-riscv \cite{RISCVSailModel2023} were the most viable choices. Both of these meet many of the same requirements, but Spike was thought to be the easiest to integrate due to its maturity and existing use in the core-v-verif environment. 

The specialization project did not consider formal verification, so we will re-evaluate the choice of ISS below. In order to meet \Cref{issReq:formal} and support formal verification, the ISS should be synthesizable. Spike is written in \cpp and is not synthesizable. 
Sail-riscv, written in the Sail language, has experimental support for compilation to SystemVerilog \cite{Release17Remsproject2023}. However, this feature is still under development and not yet implemented into the sail-riscv model \cite{SupportSystemVerilogCompilation2024}.

Future work should evaluate sail-riscv for formal verification when this feature is supported, but we choose to use Spike for this thesis because of its current advantages. This highlights the importance of \Cref{issReq:replacable} to make the ISS replaceable in the future.  


%If the model could generate synthesizable SystemVerilog code, this would meet \Cref{issReq:formal}. This feature is however still very experimental, and at the time of writing not yet implemented into the sail-riscv model\footnote{\url{https://github.com/rems-project/sail/issues/420}}.
%When this feature is implemented in the future, this would be very interesting future work to implement, but for now, we choose to go with Spike.
%We will adhere to \Cref{issReq:replacable} and make the ISS replaceable, in order to make this potential switch easy in the future

\subsection{Spike in core-v-verif}

There are multiple different versions of Spike available. We have chosen the version of Spike that already exists in the core-v-verif repository \cite{openhwgroupOpenhwgroupCorevverif2023}. We name this Spike implementation \textit{CVV Spike}. This verifies the CVA6 core, which also shares the core-v-verif verification environment. This implementation offers several advantages for integration into our reference model, meeting several of the requirements for the ISS:

To meet \Cref{issReq:step}, the ISS has to be able to step through the instructions, one instruction at a time. It must also output the processor state as RVFI to comply with \Cref{issReq:state}. The normal version of Spike runs sequentially, so some requirements are necessary to do this. These are the most important modifications done to the Spike version for core-v-verif. It implements a set of DPI function calls used to step through instructions in Spike and return a simplified RVFI output \cite{openhwgroupOpenhwgroupCorevverif2023}.

This Spike implementation is still in active development. This can be disadvantageous since we can end up relying on features that may change in the future. It can however also be an advantage, since we can influence the development.

While the CVV Spike implementation offers many benefits, it still requires modifications to be compatible with the CV32E40S core, and to fit into the rest of the pipeline shell. These modifications will be discussed in the following sections.

\section{Adapting the ISS to the CV32E40S Core}
%\ref{issReq:csr} \& \ref{issReq:binaries}}

To ensure the ISS accurately models the CV32E40S core, we need to adapt it to match some core-specific features. Although the pipeline shell should contain most of the core-specific functionality, some modifications to the ISS are also necessary (\Cref{issReq:csr}). This involves core-specific CSR implementations and configuring the memory regions of the ISS to fit the core.

\subsection{Parameters}

The CVV Spike version allows passing parameters to Spike as key-value pairs using functions like \ccode{spike_set_param_str(base,string,value)}, and \ccode{spike_set_param_uint64_t(base,string,value)}\cite{openhwgroupOpenhwgroupCorevverif2023}. It also exports \ccode{spike_set_default_params(profile)}, that allows Spike to be configured to different processors. 

We can call these functions during the ISS initialization in \ccode{iss_init(...)}  to configure the necessary modifications. 

\subsection{Mismatched CSRs}%: ISS Requirement \ref{issReq:csr}}

There are discrepancies between the CSR implementations in the core and Spike, due to ambiguities in the RISC-V specification \cite{watermanRISCVInstructionSet2021}, as well as core-specific CSRs. Some modifications to CSRs are listed below:

\begin{itemize}
    \item \rv{mvendorid}: This CSR is specific to the vendor and should be reset to the OpenHW vendor ID (\lstinline{0x602}) 
    \item \rv{marchid}: This is unique to each microarchitecture from a vendor. It should be reset to CV32E40S \rv{marchid} (\lstinline{0x15})
    \item \rv{misa}: Spike does not have bit 23 set, which should be set if non-standard extensions are present\cite{watermanRISCVInstructionSet2021}. The CV32E40S hard-wires this to 1, so we add the xdummy" string to the ISA string passed to Spike. This sets bit 23 high in Spike.
    \item \rv{mcountinhibit}: CV32E40S resets this to 5 instead of 0, requiring the same modification in Spike.
    \item \rv{mcycle}: Disabling mcycle based on the value of \rv{mcountinhibit} is not implemented in Spike. To implement this, we add a check if \rv{mcountinhibit} is set before increasing the clock count.
\end{itemize}

%\subsection{\rv{dcsr}}
%\tmp{Unødvendig?}
%
%The \rv{dcsr} debug control and status CSR differs between the core and spike. The core has the \rv{MPRVEN} and \rv{STOPCYCLE} fields hardwired to 1, while these are hardwired to 0 in Spike. Both these fields are specified as optional in the RISC-V Debug Specification \cite{pauldonahueRISCVDebugSupport2023}, so these field should be configurable in spike to match different cores.
%




%Mstatus differs this is from \cite{watermanRISCVInstructionSet2021}:
%M-mode software can determine whether a privilege mode is implemented by writing that mode
%to MPP then reading it back.
%If the machine provides only U and M modes, then only a single hardware storage bit is
%required to represent either 00 or 11 in MPP.


\subsection{Memory Layout}%: ISS Requirement \ref{issReq:binaries}}

To run the same binaries as the core (\Cref{issReq:binaries}), Spike must be modified with the same memory layout as the memory layout from the core.
The binaries used in core-v-verif are built using a linker script from the \acrfull{bsp} provided with core-v-verif. This linker script defines the memory layout of the program.


The core-v-verif linker script \cite{openhwgroupCorevverifCv32e40sBsp} has two sections, \lstinline{ram} and \lstinline{dbg}. The ram starts at address \lstinline{0x0} and has a size of \lstinline{0x400000}. The debug section starts at \lstinline{0x1A110800} and has a size of \lstinline{0x1000}. Additionally, the core has memory regions for virtual periphery registers  \lstinline{0x00800000}.


Spike's default configuration includes a boot ROM and a debug module that conflicts with the memory regions of the core \cite{evancoxAddDocumentationLowlevel2017}. To resolve this, we disable this boot loader and debug module and add support for configuring this with parameters. We also add parameters to enable and configure the location and size of the memory regions above.

These modifications allow an ELF binary file generated with core-v-verif to be run successfully in Spike. 
\Cref{lst:config} shows how all the modifications described in this section can easily be configured in \ccode{spike_set_default_params()} using parameters.

\begin{clisting}[caption=Code in \ccode{spike_set_default_params(profile)} configuring Spike for the CV32E40S core.,label=lst:config]
else if (strcmp(profile, "cv32e40s") == 0)
  {
    ...

    //Disable the debug module and boot ROM
    params.set_bool("/top/", "dtb_enabled", false);
    params.set_bool("/top/", "bootrom", false);

    //Configure ram memory region
    params.set_uint64_t("/top/", "dram_base", 0x00000000UL);
    params.set_uint64_t("/top/", "dram_size", 0x400000UL);

    //Configure the debug memory region
    params.set_bool("/top/", "dbg", true);
    params.set_uint64_t("/top/", "dbg_base", 0x1a110800UL);
    params.set_uint64_t("/top/", "dbg_size", 0x1000UL);

    //Configure the memory region for virtual peripherals
    params.set_bool("/top/", "vp", true);
    params.set_uint64_t("/top/", "vp_base", 0x00800000UL);
    params.set_uint64_t("/top/", "vp_size", 0x1000UL);

    //Configure core-specific CSRs
    params.set_uint64_t("/top/core/0/", "marchid", 0x15UL);
    params.set_uint64_t("/top/core/0/", "mvendorid", 0x602UL);
    params.set_uint64_t("/top/core/0/", "mcountinhibit", 0x5UL);

    params.set_uint64_t("/top/core/0/", "num_prev_states_stored", 4UL);
  } 
\end{clisting}

%\section{Modifications to interrupt taking: ISS Requirement \ref{issReq:interruptEnabled} \& }


%\label{sec:iss_interruptModifications}
%
%\subsection{Interrupt allowed}
%
%To support \Cref{issReq:interruptEnabled}, we add a separate \ccode{interrupt_allowed} variable to the \ccode{processor_t} class that is checked before taking an interrupt. 
%
%\subsubsection{External interrupts}
%
%The CV32E40S core only has some of the interrupt pins enabled, and some pulled to 0\cite{openhwgroupExceptionsInterruptsCOREV2023}. For the standard interrupts in bit \texttt{15-0} in \rv{mip}, these are specified in the RISC-V specification \cite{watermanRISCVInstructionSet2021} and will be set correctly in Spike if it is configured with the correct privilege levels.
%The problem is with the external interrupts in bit \texttt{31-16} of \rv{mip}, which are not enabled in Spike. We must modify the write mask used when Spike writes to \rv{mie} to enable the custom interrupts by setting bit \texttt{31-16} high.
%
%\subsection{Interrupt Ordering}
%


\section{ISS Modifications for Interaction With the Pipeline Shell}

To integrate the ISS into the reference model, we need to modify Spike to support the interface introduced in \Cref{sec:des_iss_interface}. We must be able to step through individual instructions and return RVFI data with the \sv{iss_step()} function. Interrupts and debug requests must be passed with \sv{iss_intr()} and \sv{iss_set_debug()}, and we must be able to revert the state of the ISS to a previous state. Adhering to this interface also makes the ISS easy to replace, fulfilling \Cref{issReq:replacable}.
This will be covered in this section.

\subsection{Step Through Individual Instructions}%: ISS Requirement \ref{issReq:step} }

The CVV Spike implementation already supports single-stepping through instructions using DPI functions defined in \ccode{riscv_dpi.cc} \cite{CorevverifVendorRiscv}.

The \ccode{Processor::step()} is the key method that fulfills both \Cref{issReq:step} and \ref{issReq:state}, stepping an instruction, populating the \ccode{st_rvfi} struct, and returning this to the pipeline shell over \acrshort{dpi}. 

\subsection{Return RVFI After Instruction}%: ISS Requirement \ref{issReq:state}}


To meet \Cref{issReq:state}, we must return RVFI after each instruction. The previously implemented RVFI output in CVV Spike is too primitive for our needs and does not match the RVFI output from the CV32E40S core. This requires modifications to the \ccode{Processor::step()} method. Some of these problems and modifications will be briefly explained below, but the exact modifications can be found in \Cref{app:spike_step}.


\begin{itemize}
    \item \textbf{Incorrect \rv{rs2} reporting}
    \par CVV Spike reports the wrong \rv{rs2} values for instructions that do not have an \rv{rs2} field since it reads the \rv{rs2} value directly from the instruction bits without checking the instruction type. The instruction type should be checked before reading these bits to output these values properly.
    
    \item \textbf{Missing Memory Output}
    \par CVV Spike does not implement the RVFI memory outputs in the \mbox{\rv{rvfi_mem_...}} RVFI signals. We support this by utilizing Spikes commit logs, which store register and memory writes, to generate the \rv{rd} and memory RVFI signals.


      %\rv{rvfi_mem_addr}
      %\rv{rvfi_mem_rdata}
      %\rv{rvfi_mem_rmask}
      %\rv{rvfi_mem_wdata}
      %\rv{rvfi_mem_wmask}

    
    \item \textbf{Interrupt and Trap Output}
    \par The \rv{rvfi_intr} signal must be implemented, and the \rv{rvfi_trap} signals must be modified to report interrupts and exceptions out from the core correctly. Since the \rv{rvfi_intr} signal must be reported on the first instruction of the interrupt handler \cite{openhwgroupRISCVFormalInterface2023}, we delay the \rv{rvfi_intr} signal one step to time it correctly.
    
    \item \textbf{Debug mode Output}
    \par CVV Spike does not implement the RVFI debug signals, which we must also implement. \rv{rvfi_dbg_mode} must be high for all instructions run in debug mode\cite{openhwgroupRISCVFormalInterface2023}, so we use the internal \ccode{debug_mode} state variable. 

    \item \textbf{CSR Output}
    \par To meet \Cref{issReq:CSR_out}, we must output CSR changes over RVFI. This also requires some modifications to the step function. In Spike CSR writes are also stored in a commit log, and can be used to output the modified CSR values.
    
\end{itemize}



%\subsection{RVFI problems}
%RVFI from CVA6 is sometimes off. RVFI signals are generated by calling \lstinline{insn.rs1()} which selects the correct number of bits from the insn. This is a problem when the RISC-V format does not include this field. For example the insn \lstinline{addi x3, x3, 1224} is an I-type instruction, with only rs1 and rd registers, and not rs2. When RVFI items are generated and \lstinline{insn. rs2()} is called, it does not consider the instruction type and incorrectly stores the rs2 bits that in the case of an I-type overlap with the Imm bits, leading to an incorrect rs2 rvfi item.
%
%
%\subsection{Adding the memory section to rvfi}
%
%Another modification involves adding support 
%
%CVV Spike does not support memory output over RVFI, so we will add this support. 
%CV32E40S has expanded RVFI to support multiple memory operations per instruction \cite{openhwgroupRISCVFormalInterface2023}. Since all RV32I instructions support single-memory operations, we will implement this first.
%
%For logging, spike stores all register and memory commits during an instruction. This can be used to generate the \rv{rd} and memory RVFI signals. Register commits are stored in the unordered map \ccode{log_reg_write} with register number and data, while memory read and write commits are stored in the vectors \ccode{log_mem_read} and \ccode{log_mem_write} containing tuples address, value, and size.
%
%We use the values stored in these \textit{commit logs} to generate the RVFI outputs. The details can be found in \ref{}
%
%\subsection{RVFI interrupts and traps}
%
%Some modifications are also required for the interrupt signals in RVFI.
%
%In Spike, interrupts and synchronous traps are both reported using the \ccode{which_trap} variable, but interrupts are marked with the MSB high.
%
%For exceptions (traps), the \lstinline{rvfi_trap} signal is asserted the instruction the trap occurs, while \lstinline{rvfi_intr} is asserted on the next instruction, which is the first instruction of the trap handler. To delay the interrupt RVFI output one step, we store the interrupt output into \ccode{next_rvfi_intr}, which is output to \ccode{rvfi_intr} at the next instruction.
%
%
%%For interrupts, we should only output \lstinline{rvfi_intr} and not \lstinline{rvfi_trap}. When taking an interrupt, Spike must call the step function two times, one time to take the interrupt, and the second to actually execute the first instruction of the interrupt handler. In spike, the information about the interrupt is generated during the first step and \lstinline{taken_trap} stores if an interrupt or trap happens, and \lstinline{which_trap} holds the cause of the trap/interrupt. Aditionally the MSB of signals wether the trap is for an exception or interrupt. Since we only collect the rvfi data from the second step when taking interrupts, we generate the \lstinline{rvfi_intr} signals in the first step when the trap data is available, and store this to \lstinline{next_rvfi_intr}, which is output to \lstinline{rvfi_intr} at the next step, along with the interrupt handler.
%
%\subsection{Debug}
%
%Similarly to interrupts, a delay must also be added to debug mode caused by \rv{ebreak}. This should report a trap on the instruction \rv{ebreak} is executed and output to \sv{rvfi_dbg} when at the next instruction when debug mode is active.
%
%\rv{rvfi_dbg_mode} should be high for all instructions run in debug mode. We can use the \ccode{debug_mode} stored in the \ccode{state_t} to generate this but also delay this one instruction by reading from the previous state to only output it when the whole instruction ran in debug mode, and not when an instruction enabled debug mode.
%
%\subsection{CSR}
%
%To support \Cref{issReq:CSR_out}, we also want to output CSR changes over RVFI. This is currently also implemented in the CVV Spike version.

\subsection{Interrupt Injection}%: ISS Requirement \ref{issReq:interrupt}, \ref{issReq:interruptOrdering}, \ref{issReq:set_CSR}, and \ref{issReq:interruptEnabled}}
\label{sec:iss_interruptInjection}

%This function is called every clock cycle to determine if an interrupt can be taken. 
To inject interrupts into Spike, we introduce a new \acrshort{dpi} function called \sv{iss_intr()}, which calls the \ccode{Processor:interrupt(...)} method in Spike shown in \Cref{lst:interrupt}. This is an important method for this design and will be covered in detail here.
%This is an essential method in the design and fulfills ISS Requirement \Cref{issReq:interrupt}, \ref{issReq:set_CSR}, and \ref{issReq:interruptEnabled}.
The main responsibility of the method is to set the \rv{mip} bits to reflect the incoming interrupt, informing Spike of the interrupt, and fulfilling \Cref{issReq:interrupt}. However, the previous chapter discusses multiple challenges when injecting interrupts into the ISS, which the method must also consider.

As discussed in \Cref{sec:ps_interrupt}, the ISS should be responsible for determining whether an interrupt is taken. It does, however, need multiple signals from the pipeline shell to correctly make this decision.

\Cref{sec:ps_interrupt_to_iss} determined that we need to inject the \sv{interrupt_allowed} signal into the ISS for it to properly decide if an interrupt should be taken. To fulfill this, the \ccode{interrupt()} method takes in an \ccode{interrupt_allowed} parameter. We introduce a new local state variable to Spike called \ccode{interrupt_allowed} that is checked every time Spike attempts to take a new interrupt. This allows interrupts to be enabled and disabled independently of the \rv{mip} CSRs, fulfilling \Cref{issReq:interruptEnabled}.

As discussed in \Cref{sec:ps_side-effects}, some side-effects also affect the timing of interrupts. To fulfill \Cref{issReq:set_CSR}, we must apply these CSRs from the external CSR into Spike to properly time interrupts. We do this by adding \rv{mie} and \rv{interrupt_allowed} as parameters to the interrupt method. We set the internal \rv{mie} value from this parameter and \rv{mstatus.mie} discussed in \Cref{sec:ps_side-effects} is applied to the \sv{interrupt_allowed} signal before being passed to Spike.

\Cref{sec:ps_revertion} also discussed how the ISS needs to revert its state before taking an interrupt, but after deciding that it must be taken. Because of this, the interrupt method also takes the number of instructions to revert as a parameter and reverts the state if it decides an interrupt should be taken. State reversion in the ISS is discussed further in \Cref{sec:iss_revert}.


The method is shown in \Cref{lst:interrupt}. This shows that it starts by setting the \sv{interrupt_allowed} and \rv{mie} values in line \ref{line:interrupt_allowed} and line \ref{line:mie}.
It then decides if the interrupt will be taken in line \ref{line:will_trigger}, using the \sv{interrupt_allowed} signal and the \ccode{will_trigger_interrupt(mip)} method. This checks if the interrupt will be taken, given the internal state and CSRs of Spike.

If the interrupt is taken, we follow the order of events from \Cref{sec:ps_when_revert}, first reverting the state in line \ref{line:revert}, then setting the \rv{mip} bits in line \ref{line:mip}, and stepping the ISS once in line \ref{line:step}.

When Spike takes an interrupt, the step is used to change the state of the processor, storing the previous PC to \rv{mepc} and changing the PC to the correct interrupt handler, but it does not actually execute the instruction at the new PC. For an interrupt to be taken and the first instruction of the interrupt handler to be run, we must step Spike two times. Since we export separate interrupt and step functions, we choose to step once in the interrupt function, but only if we know the interrupt will be taken. This ensures that the next time the \rv{iss_step()} function is run, it will actually execute an instruction instead of only taking the interrupt. 


We also set the \rv{mip} bits even if the interrupt is not taken in line \ref{line:mip_not}. This is to avoid a mismatch if the the \rv{mip} CSR is read while interrupts are not enabled or allowed.




\begin{clisting}[label=lst:interrupt, caption=The function used to apply interrupt in spike.,escapechar=|]
bool Processor::interrupt(reg_t mip, reg_t mie, uint32_t revert_steps, bool interrupt_allowed) {
  this->interrupt_allowed = interrupt_allowed; |\label{line:interrupt_allowed}|
  state->mie->write_with_mask(IRQ_MASK, mie); |\label{line:mie}|
  if(interrupt_allowed && will_trigger_interrupt(mip)) { |\label{line:will_trigger}|
    this->revert_step(revert_steps); |\label{line:revert}|
    state->mip->write_with_mask(IRQ_MASK, mip); |\label{line:mip}|
    this->step(1, vref); |\label{line:step}|
    return true;
  } else {
    state->mip->write_with_mask(mask, mip); |\label{line:mip_not}|
    return false;
  }
} 
\end{clisting}

The RISC-V specification specifies the priority order of the standard interrupt bits (bit \texttt{0-15}) in \rv{mip} but does not specify a priority order for the custom external interrupts in bit \texttt{31-16} \cite{watermanRISCVInstructionSet2021}, as these can be platform specific. In CV32E40S, the external interrupt with the highest ID will get the highest priority \cite{openhwgroupExceptionsInterruptsCOREV2023}.


\Cref{issReq:interruptOrdering} specifies that the ISS must be able to decide which interrupt to take if multiple are enabled. Therefore, we must also modify Spike to add external interrupts and the correct prioritization.
To make the interrupt priorities the same in the core and Spike, we can modify the functionality of the \ccode{processor_t::take_interrupt()} method to correctly order the external interrupts. 
This approach assumes that the interrupt ordering for the external interrupts will be the same for all cores. This might not necessarily be the case since the ordering is not defined in the specification \cite{watermanRISCVInstructionSet2021}. Future work should, therefore, make the ordering of external interrupts configurable or move the interrupt-taking functionality of Spike to a separate, core-specific module.






\subsection{State Reversion}%: ISS Requirement \ref{issReq:revertState}}
\label{sec:iss_revert}

\Cref{issReq:revertState} specifies that we must be able to revert Spike to a previous state to discard instructions that are flushed from the pipeline shell, as discussed in \Cref{sec:ps_revertion}. This involves restoring the \acrshort{pc}, register file values, \acrshort{csr}s, memory contents,  and other relevant state information. 

We never have to store more states than the number of stages in the pipeline because revertion is only done to discard instructions flushed from the pipeline. This allows a fixed-size array to be used with the same size as the number of pipeline stages. If a synthesizable ISS is used in the future, using a fixed-size array would also make these modifications synthesizable, meeting \Cref{issReq:formal}.

Most of this state data is encapsulated in Spikes \ccode{state_t} struct. A straightforward approach is to store a snapshot of this struct before executing each instruction. When a flush occurs, we can restore a previous state by replacing the current state with a previous snapshot state. 

Using \ccode{sizeof(state_t)}, we find that the struct size is 3592 bytes. While storage constraints are not a primary concern, future work could be to store only the relevant data to a smaller struct. For now, we can accept the size of  \ccode{state_t}. 

Memory writes during an instruction gets more complicated, as these are not stored in the \ccode{state_t} struct. There are several possible approaches:

\begin{enumerate}
    \item \textbf{Full memory snapshots}: One solution could be to store a complete memory copy before each instruction. Spike uses a \ccode{sparse_memory_map} to store locations of different memory pages \cite{SpikeRISCVISA2023}. To copy the whole memory we would have to make a copy of each page pointed to by the memory map. Additionally, the cache and TLB also have to be properly updated. This is highly impractical due to the potentially large memory size and unnecessary complexity.
    \item \textbf{Cache snapshots}: Another solution is to store a snapshot of the cache. The cache is smaller than the whole memory, but restoring it might not be as straightforward since memory writes can bypass the cache and store it straight to memory \cite{hennessyComputerArchitectureQuantitative2019}.
    \item \textbf{Storing pre-write values}: Another approach is to store the original values at memory locations before these are written to. On state reversion, we can iterate over the stored values and store them back using the normal memory interface, avoiding considering the cache or TLB.
\end{enumerate}


%\subsubsection{Storing the changes to memory for every instruction. (or values before memory operation)}

We choose the last option, storing the values before a write is done to the address. This approach is drastically smaller and simpler than the other approaches.
When the pipeline is flushed, each memory operation must be reverted to its original state. This allows us to use the top-level \ccode{store()} function in Spike to revert each memory operation. This would automatically handle the TLB, caches, and storing to the correct memory address on the correct page.

The \ccode{state_t} already contains commit logs holding reads and writes to registers and memory, used to make the execution log from Spike. The memory logs hold address, value, and size in a vector of tuples like \ccode{typedef std::vector<std::tuple<reg_t, uint64_t, uint8_t>> commit_log_mem_t;}. The commit log is written to in \ccode{store()} in \file{mmu.h}. We are interested in the value before writing, and not the value stored in the commit log, which is the value after writing. 

We can therefore expand the \ccode{state_t} to hold another commit log of values before writing, called \ccode{log_mem_pre_write}. This vector holds the address, value, and size of memory operations. We can expand the \ccode{store()} function in \file{mmu.h} to read the value of the given address before writing to it and storing this value and the address to \ccode{log_mem_pre_write}.

%To load the memory values before writing to them, we can either use the \ccode{load()} function from inside \ccode{store()} or directly use the relevant lines from \ccode{load()} in \ccode{store()}. If we were to call the \ccode{load()} function, this would also store to the \ccode{log_mem_read} commit log, which would affect the generated RVFI output and spike execution log. We would therefore have to clear the \ccode{log_mem_read} after running \ccode{load()} or disable the commit log before running it.

%Since the address generation from \ccode{load()} and \ccode{store()} are similar, we can easily add the relevant lines of code from \ccode{load()} to \ccode{store()}. The expanded \ccode{store()} from \file{mmu.h} is shown in \Cref{lst:store} with the added lines marked with \ccode{+}.
%
%
%\begin{clisting}[caption={Modified \ccode{store()} function from \file{mmu.h} with new lines marked with \ccode{+}.}, label={lst:store}]
%  template<typename T>
%  void ALWAYS_INLINE store(reg_t addr, T val, uint32_t xlate_flags = 0) {
%    reg_t vpn = addr >> PGSHIFT;
%    bool aligned = (addr & (sizeof(T) - 1)) == 0;
%    bool tlb_hit = tlb_store_tag[vpn % TLB_ENTRIES] == vpn;
%    target_endian<T> previous_value;
%
%
%    if (xlate_flags == 0 && likely(aligned && tlb_hit)) {
%+     //Store previous value before writing
%+     previous_value = *(target_endian<T>*)(tlb_data[vpn % TLB_ENTRIES].host_offset + addr);
%
%      *(target_endian<T>*)(tlb_data[vpn % TLB_ENTRIES].host_offset + addr) = to_target(val);
%    } else {
%+     //Store previous value before writing
%+     load_slow_path(addr, sizeof(T), (uint8_t*)&previous_value, xlate_flags);
%
%      target_endian<T> target_val = to_target(val);
%      store_slow_path(addr, sizeof(T), (const uint8_t*)&target_val, xlate_flags, true, false);
%    }
%
%    if (unlikely(proc && proc->get_log_commits_enabled())) {
%      proc->state.log_mem_write.push_back(std::make_tuple(addr, val, sizeof(T)));
%+     proc->state.log_mem_pre_write.push_back(std::make_tuple(addr, from_target(previous_value), sizeof(T)));
%    }
%  }
%\end{clisting}

\subsubsection{Storing States for Instructions}

Since we store the \ccode{log_mem_pre_write} in the \ccode{state_t}, it should be sufficient to only store this state struct for each instruction in the pipeline to store the registers, \acrshort{csr}s, and memory operation for each instruction. 

To avoid converting this struct to SystemVerilog and back to \cpp, it is easier to store this state \textit{snapshot} in Spike and use a DPI function call from the pipeline shell to signal which snapshot to revert back to.

In Spike, we can store the previous states in a \ccode{deque} double-ended queue. This makes it easy to insert the most recent state at the beginning of the queue, and remove the oldest state from the end of the queue when the instruction leaves the pipeline. 
We can also easily access items at a specific index to revert the state to a given previous state.

We store the previous states in the \ccode{Processor} class. We store the previous state at the beginning of the \ccode{step()} function. After executing the instruction, we store the \ccode{llog_mem_pre_write} vector into a \ccode{prev_changes_t} struct along with the previous state. This is stored in the \ccode{previous_states} \ccode{deque}.

\subsubsection{Reverting the State}

We introduce the \ccode{Processor::revert_step(int num_steps)} method to revert the state.
To revert the state, we override the current state with the state stored at the index of \ccode{num_steps} in \ccode{prev_changes}. To revert the memory operations, we iterate over all the previous states and the content of \ccode{log_mem_pre_write}. For all the stored memory writes, we call the \ccode{store()} method with the stored address, value, and size of each memory write.

To get the order of operations right, the \ccode{revert_step()} method is called inside the \ccode{interrupt()} and \ccode{debug()} methods, as explained in \Cref{sec:ps_revertion} and \Cref{sec:iss_interruptInjection}.


%\section{Custom instructions:\Cref{issReq:custom}}
%
%\tmp{ZCMP supports multiple memory operations per instruction. This affects RVFI}
%
%\tmp{Also supports multiple RD writes, which must also be handled with RVFI}

%\section{Update: \Cref{issReq:update}}


%%A \rv{wfi} should as wake up to an interrupt even though \rv{mstatus.mie} is disabled \cite{watermanRISCVInstructionSet2021}. 
%%The added \ccode{interrupt_allowed} bool added to Spike should not stop a \rv{wfi}we added should not block \ccode{take_interrupt(pending_interrupt)} from running even if interrupts are not allowed, since this could lead to a \rv{wfi} not waking up when an interrupt is passed to spike.


%To allow the pipeline to fill up with the upcoming instructions after the \rv{wfi}, we can disable wfi in Spike by setting \ccode{in_wfi = false} in the step function in \file{Proc.cc}. This way, the next instruction will run on the next step, and we will have to make sure \rv{wfi} waits operate as intended outside of Spike in the reference model. When using the retirement dependent approach, the \rv{wfi} instruction will automatically wait to be run until the core runs, but we need some way to determine that an interrupt actually occurred, causing the execution to continue after the \rv{wfi}. \tmp{TODO: this}

\section{Debug Requests}


To insert asynchronous interrupts into Spike, we define a new DPI function \ccode{iss_set_debug(...)} that calls the method \ccode{Processor::set_debug(bool debug_req, uint32_t revert_steps, bool debug_allowed}. It takes in the debug request, the number of revert steps, and the \ccode{debug_allowed} as parameters, similarly to \ccode{iss_intr(...)}. 

If Spike is not in debug mode, a new debug request is made, and if debug is allowed, we set the \ccode{processor_t::halt_request} variable to \ccode{HR_REGULAR}. At the next step, Spike will enter debug mode, store the cause and privilege to \rv{dcsr}, store the PC to \rv{dpc}, and change the PC to the debug handler.

One timing difference between interrupts and debug requests in the CV32E40S core is that when an interrupt is taken, the pipeline is immediately flushed, and the PC changed to the interrupt handler, while when an async debug is taken, the pipeline is first halted one cycle, before flushing all or parts of the pipeline the next cycle. The WB stage is also not flushed if the interrupt is caused by \rv{EBREAK} or \rv{TRIGGER}.



