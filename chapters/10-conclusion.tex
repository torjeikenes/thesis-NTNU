\chapter{Conclusions and Future Work}
\label{ch:conclusion}

\tmp{Er fortsatt litt WIP}


This thesis focused on designing and implementing a reference model for RISC-V processor cores that can accurately simulate asynchronous events. We explored the challenges of verifying asynchronous events and the limitations of using an \acrfull{iss} as a reference model. To address these limitations, we proposed a reference model architecture that combines a pipeline shell on top of an existing ISS. 
The pipeline shell is responsible for modeling the timing and behaviour of the pipeline, while the ISS is responsible for the functional execution of the instructions.



\tmp{TODO: mer konkludering}

\section{Future Work}
\tmp{TODO}

\subsection{Implement the \sv{interrupt_allowed} signal}

As the \sv{interrupt_allowed} signal is a vital part of the proposed design, it is important to further evaluate if it is possible to model the signal based on the information in the pipeline shell.


\subsection{Using state revertion to implement a state exploration approach}

The state revertion implementatio explained in \Cref{sec:ps_revertion} and \Cref{sec:iss_revert} has the potential to be used to implement state exploration approach similar to ImperasDV. Given a "fork" of different operations, it could attempt one possible state, compare with the core, and revert the state if there is a mismatch. It can then attempt the next possible state change, and revert if this also has a mismatch. If all state changes cause a mismatch, this can be flagged as an error.


\subsection{Replace the ISS with sail-riscv}

Work is currently being done to implement SystemVerilog compilation of the sail-riscv model\footnote{https://github.com/riscv/sail-riscv/issues/424}. When this is completed, it would be interesting to replace Spike with sail-riscv, and evaluate wether this approach could work with formal verification. 

\subsection{Improvements based on limitations}

\tmp{Diskuterer litt løsninger i Limitations i diskusjonen. Bør jeg flytte de hit, eller bare nevne at de er i diskusjonen?}