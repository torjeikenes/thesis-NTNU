\chapter{Formal Verification}
\label{ch:formal}

\section{core-v-verif Onespin integration}

In order to support formal verification, one contribution of this thesis is a script that configures Onespin \cite{onespinsolutionsgmbhUserManualOneSpin} to support the CV32E40S core, as well as makefile modifications to properly run the testbench. This contribution is the addition of the file \file{onespin.tcl} found in \Cref{app:onespin.tcl}. This loads the correct files and configures the core for formal verification in Onespin. 

Onespin needs signals to drive in order to test different inputs. By default, these signals are already driven in the testbench, so these need to be disconnected or \textit{cut} for the signals to show up as inputs and be drivable by onespin\cite{ReferenceManualOneSpin}. This can be done with the \lstinline{add_compile_option -cut_signal clknrst_if/clk}.

To get Onespin set up, we cut the clock (\sv{clk}) and reset (\sv{reset}) signals so Onespin can drive them. Additionally, to test different instructions, values can be injected into the \acrfull{obi} \cite{siliconlabsincOBI} used to read instructions from memory by cutting the signals in \lstinline{obi_instr_if} and \lstinline{obi_data_if}.


\section{Formal Verification of the Reference Model}
\label{sec:formal_testbench}
\label{sec:formal_dummy}

The previous section covered how the standard core-v-verif environment can be configured to work with Onespin. This focuses on using Onespin to verify the reference model itself. 

When running the makefile controlling core-v-verif, the reference model can also be enabled and disabled with a set of defines. We need to support these defines in our solution to support the reference model in onespin.

Since Spike is written in \cpp, it is incompatible with formal verification. To support formal verification in the future by switching to a formal-friendly ISS, we want the rest of the reference model to be compatible with formal verification. In addition to the simulation-based testbench, we want to verify the design using formal methods. To do this, we replace \lstinline{iss_wrap_pkg.sv} that communicates with Spike, with a "dummy ISS " in \lstinline{iss_wrap_formal_pkg.sv} that is mostly empty but can be used to verify that the rest of the reference model works with formal methods.


For the dummy ISS interface to be used by the rest of the reference model, it should have the same interface and return \acrshort{rvfi} signals. Since we do not have a working formal ISS, we can either output a static \acrshort{rvfi} item or return the same \acrshort{rvfi} signals as the core. We can also verify that the comparison functionality works by returning the same RVFI signals as the core.


We would then have two environments: a UVM simulation environment that verifies the correctness of the reference model using a \cpp \acrshort{iss}, and a formal verification environment that does not test the functionality of the reference model, but checks that the reference model can run in a formal verification environment when using an RTL simulator.

If both these tests pass, future work should be to replace the dummy ISS in the formal environment with sail-riscv when the SystemVerilog generator is fully supported and verify that this works. 


\section{Benefits of Onespin Support}

The \file{onespin.tcl} script enables the use of Onespin for formal verification of the CV32E40S core, even without the reference model. The CV32E40S core contains many \acrfull{sva} for verifying properties within the core \cite{openhwgroupCv32e40s2024}. This contribution also allows these assertions to be verified using formal verification with Onespin.

Although core-v-verif already supports other formal verification tools like Jasper Formal \cite{JasperFormalVerification} and Questa Formal \cite{QuestaFormalVerification}, adding Onespin support can still be valuable. Core-v-verif is used by multiple cores and various teams with access to different tools. Supporting a wider range of tools enhances the accessibility and usability of the verification environment, enabling more users to verify RISC-V processors formally.

Adding support for a reference model with formal verification can also be very valuable. If a formal-friendly ISS can be used in the future, making the proposed reference model compatible with formal verification, this has many benefits explained in \Cref{sec:bg_formal}. By doing this, we can mathematically prove that the core and reference model will operate the same way, instead of simulation-based testing used today, where many tests are run hoping that they cover every scenario and corner case.