\chapter{Formal Verification}
\label{ch:formal}

\tmp{very WIP}

\section{core-v-verif Onespin integration}

\file{onespin.tcl} loads the correct files and configures onespin. 


Onespin needs signals to drive in order to test different inputs. By default, these signals are already driven in the testbench, so these need to be disconnected or \textit{cut} for the signals to show up as inputs and be drivable by onespin. This can be done with the \lstinline{add_compile_option -cut_signal clknrst_if/clk}.

To get onespin setup, onespin needs a clock and reset signal to drive. We therefore cut \sv{clk} and \sv{re}


To test different instructions, we can inject values into the obi interface by cutting the signals in \lstinline{obi_instr_if} and \lstinline{obi_data_if}.


\subsection{problems implementing scriptware}


\begin{enumerate}


\item \textbf{Issue:} \lstinline{vlog -sv -f fv.flist} fails with comments
  \begin{description}
    \item[Cause:] The vlog command does not detect comments properly
    \item[Resolution:] Remove comments from \file{fv.flist}
  \end{description}

\item \textbf{Issue:} \lstinline{vlog -sv -f fv.flist} has no way (known by author) of specifying the sv2012 version, leading to errors from sv2012 specific functionality.
  \begin{description}
    \item[Cause:] 
    \item[Resolution:] Manually add all files and includes from fv.flist with \lstinline{add_read_hdl_option -verilog_include_path <path>} and \lstinline{read_verilog -version sv2012 <file.sv>}:
  \end{description}

\item \textbf{Issue:}  \lstinline{uvm_error} not defined although it is defined in \file{uvm_pgk.sv}
  \begin{description}
    \item[Cause:] The \lstinline{vlog} command uses \lstinline{set_read_hdl_option} instead of  \lstinline{add_read_hdl_option}, which overrides previous options. The include path is therefore overwritten by \lstinline{set_read_hdl_option -verilog_compilation_unit one}.
    \item[Resolution:] Dont use \lstinline{vlog}, but manually add files.
  \end{description}
  
 \item \textbf{Issue:} Error: \lstinline{-E- UnsuppVerilog - /home/torjene/core-v-verif/core-v-cores/cv32e40s/bhv/cv32e40s_wrapper.sv:218: bind construct with instances are not supported.}
  \begin{description}
    \item[Cause:] from onespin reference manual: Binding to individual instances (\lstinline{bind_target_instance} and hierarchical instance names) is supported only when used with the \lstinline{read_sva} command, i.e. not in \lstinline{read_verilog}.
    \item[Resolution:] remove individual instances or find a way to add the assertions with \lstinline{read_sva}
  \end{description}
  
\item \textbf{Issue:} 
  \begin{description}
    \item[Cause:] 
    \item[Resolution:] 
  \end{description}
  
\item \textbf{Issue:}  \lstinline{-E- NoModule - cv32e40s_wrapper.sv:934 no module 'cv32e40s_core' for instance 'core_i' found.}
  \begin{description}
    \item[Situation:] Only loading \lstinline{cv32e40s_wrapper.sv} with necessary files.
    \item[Cause:] 
    \item[Resolution:] 
  \end{description}
 
 

\item \textbf{Issue:} \lstinline{Unsupported clocking "posedge clk_i or negedge rst_ni" for smapled value function.} This happens when  or \lstinline{$past} or \lstinline{$changed} is in an \lstinline{always_ff} outside of assertions.
  \begin{description}
    \item[Situation:] when adding all files again to \lstinline{read_sva}
    \item[Cause:] ??
    \item[Resolution:] comment out 
  \end{description}
  
\item \textbf{Issue:} \lstinline{-E- AlreadyDecl - /home/torjene/core-v-verif/core-v-cores/cv32e40s/bhv/cv32e40s_wrapper.sv:490: identifier 'mpu_if_sva' already declared at /home/torjene/core-v-verif/core-v-cores/cv32e40s/bhv/cv32e40s_wrapper.sv:490:.}
  \begin{description}
    \item[Situation:] after commenting out "unsupported clocking" errors above
    \item[Cause:] ??
    \item[Resolution:] 
  \end{description}

  
\end{enumerate}

bind error solutions??

only use \lstinline{read_sva} on \lstinline{cv32e40s_wrapper} -> \lstinline{-E- NoModule - cv32e40s_wrapper.sv:934 no module 'cv32e40s_core' for instance 'core_i' found.}

Add all files again with \lstinline{read_sva}


\section{Reference model considerations for formal verification}

Assertion based comparison

Must be synthesizable

Avoid dynamic structures

\section{Formal verification testbench}

Since spike is written in C++, it is not compatible with formal verification. In order to support formal verification in the future by switching to a formal friendly ISS, we want the rest of the reference model to be compatible with formal verification. In addition to the simulation based testbench, we also want to verify the design with formal methods. To do this, we create a new version of \lstinline{iss_wrap_pkg.sv} that communicates with Spike, with \lstinline{iss_wrap_formal_pkg.sv} that is a mostly empty, but can be used to verify that the rest of the reference model works with formal methods.

In order for the dummy ISS interface to be used by the rest of the reference model, it should have the same interface, and return \acrshort{rvfi} signals. Since we do not have a working formal ISS, we can either output a static \acrshort{rvfi} item, or return the same \acrshort{rvfi} signals as the core. By returning the same RVFI signals as the core, we can also verify that the comparison functionality works.


To verify that the  In order to verify that the rest of the reference model is compatible with formal verification.

