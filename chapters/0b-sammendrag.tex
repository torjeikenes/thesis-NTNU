\chapter*{Sammendrag}

Nøye verifikasjon er avgjørende ved utviklig av prosessorkjerner og tar en betydelig del av den totale utviklingstiden.
Asynkrone hendelser som "interrupts" er en stor utfordring i prosessorverifisering, som ofte fører til vanskelige feil. \textit{InstruksjonssettSimulatorer} (ISS) brukes ofte som referansemodell for å verifisere prosessorkjerner, og er ofte tilstrekkelig for normal gjennomkjøring av prosessoren. På grunn av deres høye abstraksjonsnivå kan imidlertid ikke ISS-er nøyaktig simulere timingen og virkningen av asynkrone hendelser på samme måte som en prosessor\-kjerne. Det er derfor behov for en mer avansert referansemodell som nøyaktig simulerer timingen og korrekt\-heten av asynkrone hendelser. 

I denne masteroppgaven utforsker og sammenligner vi ulike tilnærminger til implementeringen av en referansemodell for åpne RISC-V prosessorkjerner som kan simulere asynkrone hendelser på syklusnivå.  Vi foreslår og implementerer en referansemodellarkitektur som kombinerer et “Pipeline-Skall” og en eksisterende ISS. Pipeline-skallet er ansvarlig for å modellere timingen til pipelinen og asynkrone hendelser, mens ISSen er ansvarlig for den funksjonelle utførelsen av instruksjonene. Referansemodellen er implementert i SystemVerilog og integrert i OpenHW Groups verifiseringsmiljø for CV32E40S-kjernen til Silicon Labs. Spike er valgt som den eksisterende ISSen, og de nødvendige modifikajonene blir diskutert. I tillegg fokuserer vi på å gjøre referansemodellen kompatibel med formell verifisering.

Resultatene viser at referansemodellen kan simulere asynkrone hendelser korrekt i mange ulike scenarier. Den har et mindre verifikasjonsgap sammenlignet med å bruke en tradisjonell ISS som referansemodell, og kan sannsynligvis finne feil som andre referansemodelløsninger ikke kan. ISSen som brukes er ikke kompatibel med formell verifikasjon, men resten av designet antas å være kompatibelt med formell verifikasjon dersom ISSen erstattes med en syntetiserbar ISS. Imidlertid har de nåværende implementeringen noen begrensninger. Den implementerte referansemodellen er avhengig av signaler fra prosessorkjernen for noen detaljer i pipelinetimingen, og kompleksiteten til pipeline-skallet har ført til flere hittil uløste feil.


