\chapter*{Sammendrag}

Asynkrone hendelser som "interrupts" er en stor utfordring i prosessorverifisering, som ofte fører til vanskelige feil. \textit{InstruksjonssettSimulatorer} (ISS) brukes ofte som referansemodell for å verifisere prosessorkjerner, og er ofte tilstrekkelig for normal gjennomkjøring av prosessoren. På grunn av deres abstraksjonsnivå på instruksjonsnivå kan imidlertid ikke ISS-er nøyaktig simulere timingen og bivirkningen av asynkrone hendelser sammenlignet med prosessorkjernen. Det er derfor behov for en mer avansert referansemodell som nøyaktig simulerer timingen og korrektheten av asynkrone hendelser. 

Denne masteroppgaven utforsker og sammenligner ulike tilnærminger til  implementeringen av en referansemodell for åpne RISC-V prosessorkjerner som kan simulere asynkrone hendelser på syklusnivå.  Vi foreslår en referansemodellarkitektur som kombinerer et “Pipeline-Skall” og en eksisterende ISS. Pipeline-skallet er ansvarlig for å modellere timingen til pipelinen og asynkrone hendelser, mens ISSen er ansvarlig for den funksjonelle utførelsen av instruksjonene. Referansemodellen er implementert i SystemVerilog og integrert i OpenHW Groups verifiseringsmiljø for CV32E40S-kjernen til Silicon Labs. I tillegg fokuserer vi på å gjøre referansemodellen kompatibel med formell verifisering.

Resultatene viser at referansemodellen kan simulere asynkrone hendelser korrekt i mange forskjellige scenarier. Den har et mindre verifikasjonsgap sammenlignet med å bruke en tradisjonell ISS som referansemodell, og kan sannsynligvis finne feil som andre referansemodelløsninger ikke kan. ISSen som brukes er ikke kompatibel med formell verifikasjon, men resten av designen antas å være kompatibelt med formell verifikasjon dersom ISSen erstattes med en syntetiserbar ISS. Imidlertid har de nåværende implementasjonen noen begrensninger. Den implementerte referansemodellen er avhengig av signaler fra prosessorkjernen for noen pipelinetimingdetaljer, og kompleksiteten til pipeline-skallet har ført til flere hittil uløste feil.


